\documentclass[a4paper,12pt]{article}
\usepackage[utf8]{inputenc}
\usepackage{graphicx}
\usepackage{amsmath}
\usepackage{geometry}
\geometry{margin=2.5cm}
\title{Étude de la modulation PWM par comparaison}
\author{TheoPec}
\date{\today}

\begin{document}
\maketitle

\section{Introduction}
La modulation de largeur d'impulsion (PWM) est une technique couramment utilisée pour contrôler la puissance délivrée à une charge. Elle consiste à comparer un signal d'entrée (ici une sinusoïde) à une onde porteuse (ici une onde triangulaire).

\section{Principe de la modulation PWM}
La PWM est obtenue en comparant à chaque instant la valeur du signal d'entrée à celle de l'onde triangulaire. Lorsque le signal d'entrée est supérieur à l'onde triangulaire, la sortie PWM est à l'état haut (1), sinon elle est à l'état bas (0).

\section{Simulation sous Python}
Pour vérifier ce principe, on simule :
\begin{itemize}
    \item une sinusoïde de fréquence $0{,}1$ Hz et d'amplitude $1$ V,
    \item une onde triangulaire de fréquence $1$ Hz et d'amplitude $1{,}2$ V.
\end{itemize}
Le code Python utilisé est le suivant :

\begin{verbatim}
import numpy as np
import matplotlib.pyplot as plt

f_sin = 0.1
A_sin = 1.0
f_tri = 1.0
A_tri = 1.2
T_obs = 30
fs = 1000
t = np.arange(0, T_obs, 1/fs)
sinus = A_sin * np.sin(2 * np.pi * f_sin * t)
tri = A_tri * (2 * np.abs(2 * ((t * f_tri) % 1) - 1) - 1)
pwm = (sinus > tri).astype(float)
plt.figure(figsize=(12, 8))
plt.subplot(3, 1, 1)
plt.plot(t, sinus, label='Sinusoïde 0,1 Hz')
plt.ylabel('Amplitude [V]')
plt.legend()
plt.grid(True)
plt.subplot(3, 1, 2)
plt.plot(t, tri, color='orange', label='Onde triangulaire 1 Hz')
plt.ylabel('Amplitude [V]')
plt.legend()
plt.grid(True)
plt.subplot(3, 1, 3)
plt.plot(t, pwm, color='green', label='PWM (sinus > triangle)')
plt.ylabel('PWM')
plt.xlabel('Temps [s]')
plt.legend()
plt.grid(True)
plt.tight_layout()
plt.show()
\end{verbatim}

\section{Résultats}
Le graphique obtenu montre :
\begin{itemize}
    \item la sinusoïde d'entrée,
    \item l'onde triangulaire porteuse,
    \item le signal PWM résultant.
\end{itemize}

\begin{figure}[h!]
    \centering
    \includegraphics[width=0.9\textwidth]{pwm_graph.png}
    \caption{Comparaison entre la sinusoïde, l'onde triangulaire et le signal PWM}
\end{figure}

\section{Conclusion}
La simulation confirme que la modulation PWM peut être réalisée par simple comparaison entre le signal d'entrée et une onde triangulaire. Ce principe est fondamental pour le contrôle de puissance dans de nombreux systèmes électroniques.

\end{document}
